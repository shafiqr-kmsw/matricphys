\documentclass{beamer}
\usetheme{Madrid}
\usepackage{graphicx}
\usepackage{booktabs}
\usepackage{amsmath}

\title[AI Tutoring in Physics]{AI-Powered Tutoring for Conceptual Clarity\\ in Energy and Momentum}
\subtitle{An Action Research Project at Sarawak Matriculation College}
\author[Shafiq R]{Shafiq Bin Rasulan \\ \texttt{bm-3542@moe-dl.matrik.edu.my}}
\institute{KMSw}
\date{Konvensyen Penyelidikan dan Inovasi Pendidikan (KonPPI-5)\\ 2025}

\begin{document}

\begin{frame}
  \titlepage
\end{frame}

\begin{frame}{Overview}
\begin{itemize}
    \item Persistent conceptual issues in energy and momentum.
    \item Action research using AI tutoring tools: ChatGPT, Gemini, DeepSeek.
    \item Implemented with 30 matriculation physics students.
    \item Based on Socratic questioning and ICAP framework.
\end{itemize}
\end{frame}

\begin{frame}{Reflection on Past Teaching Practice}
\begin{itemize}
    \item Traditional lectures didn't fix core misconceptions.
    \item Students confused momentum with mass, energy with fuel consumption.
    \item Diagnostics: <50\% understanding in energy/momentum pre-tests.
    \item Students feared giving wrong answers in class.
\end{itemize}
\end{frame}

\begin{frame}{Research Focus}
\begin{itemize}
    \item Topics: Energy and Momentum.
    \item Selected for importance, syllabus alignment, and AI tool feasibility.
    \item Pre-assessments confirmed persistent misconceptions.
    \item Goal: Address conceptual gaps with AI-mediated dialogue.
\end{itemize}
\end{frame}

\begin{frame}{Objectives}
\begin{enumerate}
    \item Improve conceptual understanding through AI dialogue tutoring.
    \item Measure learning gains using the ECMS.
    \item Evaluate student experience via Likert-based survey.
\end{enumerate}
\end{frame}

\begin{frame}{Implementation Design}
\textbf{3-Phase Learning Cycle:}
\begin{enumerate}
    \item \textbf{Pre-class:} AI-based conceptual prompts.
    \item \textbf{In-class:} Peer discussion + teacher feedback.
    \item \textbf{Post-class:} Reflection + refinement.
\end{enumerate}
\begin{itemize}
    \item Students trained in Socratic questioning and ICAP response analysis.
    \item Used only mobile devices.
\end{itemize}
\end{frame}

\begin{frame}{Learning Gains}
\begin{itemize}
    \item Pre-test average: \textbf{46.67\%}, Post-test average: \textbf{70.76\%}.
    \item Normalized gain: \textbf{42.94\%}, Effect size (Cohen's d): \textbf{1.77}.
\end{itemize}
\begin{table}[ht]
\centering
\begin{tabular}{@{}lcc@{}}
\toprule
\textbf{Topic} & \textbf{Pre (\%)} & \textbf{Post (\%)} \\
\midrule
Energy Concepts & 41.39 & 70.00 \\
Momentum Concepts & 33.33 & 67.78 \\
Inelastic Collisions & 45.56 & 73.89 \\
Impulse-Momentum & 53.33 & 77.78 \\
\bottomrule
\end{tabular}
\end{table}
\end{frame}

\begin{frame}{Student Feedback}
\begin{itemize}
    \item Misconception correction: \textbf{4.77 / 5}
    \item Explanation clarity: \textbf{4.70}
    \item Comfort with AI vs teacher: \textbf{4.57}
    \item Recommendation for AI in other topics: \textbf{4.73}
    \item Slightly lower in stimulating critical thinking: \textbf{4.50}
\end{itemize}
\end{frame}

\begin{frame}{Reflections}
\begin{itemize}
    \item AI tutors offer:
    \begin{itemize}
        \item Immediate feedback
        \item Personalization
        \item Reduced anxiety
    \end{itemize}
    \item Effective even for hard concepts like energy transfer and momentum conservation.
    \item Need better prompts to boost critical thinking.
\end{itemize}
\end{frame}

\begin{frame}{Conclusion and Future Work}
\begin{itemize}
    \item AI tutoring significantly improved conceptual clarity.
    \item Strongly endorsed by students.
    \item Future directions:
    \begin{itemize}
        \item Long-term retention
        \item Control group comparison
        \item Improved AI-human teaching balance
    \end{itemize}
\end{itemize}
\end{frame}

\end{document}
